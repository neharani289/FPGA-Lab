\documentclass[journal,12pt,twocolumn]{IEEEtran}
\usepackage[utf8]{inputenc}
\usepackage{amssymb}
\usepackage{setspace}
\usepackage{gensymb}
\singlespacing
\usepackage[mathscr]{euscript}
\usepackage{textgreek}
\usepackage{textcomp}
\usepackage{amsmath}
\numberwithin{equation}{section}
\usepackage{mathrsfs}
\usepackage{txfonts}
\usepackage{stfloats}
\usepackage{bm}
\usepackage{cite}
\usepackage{hyperref}
\usepackage{cases}
\usepackage{subfig}
\usepackage{graphicx}
\usepackage{circuitikz}
\usepackage{tikz} 
\usetikzlibrary{karnaugh}
\usepackage{listings}
\usepackage{amsfonts}
\usepackage{longtable}
\usepackage{multirow}
\usepackage{tcolorbox}
\usepackage{geometry}
\geometry{
 a4paper,
 total={170mm,257mm},
 left=20mm,
 top=20mm,
 }
\usepackage{listings}
\lstset{
frame=single, 
breaklines=true,
columns=fullflexible
}

\def\putbox#1#2#3{\makebox[0in][l]{\makebox[#1][l]{}\raisebox{\baselineskip}[0in][0in]{\raisebox{#2}[0in][0in]{#3}}}}
     \def\rightbox#1{\makebox[0in][r]{#1}}
     \def\centbox#1{\makebox[0in]{#1}}
     \def\topbox#1{\raisebox{-\baselineskip}[0in][0in]{#1}}
     \def\midbox#1{\raisebox{-0.5\baselineskip}[0in][0in]{#1}}
\vspace{3cm}

\usepackage{karnaugh-map}
\usepackage{hyperref}

\renewcommand\thesection{\arabic{section}}
\renewcommand\thesubsection{\thesection.\arabic{subsection}}
\renewcommand\thesubsubsection{\thesubsection.\arabic{subsubsection}}

\renewcommand\thesectiondis{\arabic{section}}
\renewcommand\thesubsectiondis{\thesectiondis.\arabic{subsection}}
\renewcommand\thesubsubsectiondis{\thesubsectiondis.\arabic{subsubsection}}

\title{Assignment 1}
\author{EE20MTECH14014 \\ Neha Rani}
\date{December 2021}

\begin{document}

\maketitle
Download the LaTex code from:
\begin{lstlisting}
https://github.com/neharani289/FPGA-Lab/tree/main/Assignment1
\end{lstlisting}

\section{Question}
\vspace{15pt}

[ICSE 2017 Q5 (b)]\\
State the application of Half Adder. Draw the truth table and circuit diagram for a Half Adder


\section{Solution}
\subsection{Application of Half Adder}
\begin{enumerate}
    \item Half Adder is a combinational logic circuit.
It is used for the purpose of adding two single bit numbers.\\
\item Half adder is used to make full adder as a full adder requires 3 inputs, the third input being an input carry i.e. we will be able to cascade the carry bit from one adder to the other.\\
\item The ALU (arithmetic logic circuitry) of a computer uses half adder to compute the binary addition operation on two bits.\\
\end{enumerate}

\subsection{Truth Table of Half Adder}
\space
The truth table corresponding to sum and carry for various choices of input A, B is shown below:\\
\medskip

    \centering
          \begin{tabular}{|c|c|c|c|}
            \hline
            A & B & Sum & Carry \\
            \hline
            0 & 0 & 0 & 0\\
            \hline
             0 & 1 & 1 & 0\\
            \hline
             1 & 0 & 1 & 0\\
            \hline
             1 & 1 & 0 & 1\\
            \hline
        \end{tabular}
        \vspace{100pt}
\subsection {Circuit Diagram}

\begin{figure}[h!]
    \centering
    \begin{circuitikz}\draw
       
         (0,0) node[xor port](myxor1){}
        (0,2) node[and port](myand1){}
        (-5,2.25) to[short,-*](-3,2.25)
        (-5,1.75) to[short ,-*](-4,1.75)
        (-3,2.25)-|(myand1.in 1)
        (-4,1.75)-|(myand1.in 2)
        (-3,2.25)--(-3,0.25)-|(myxor1.in 1)
        (-4,1.75)--(-4,-0.25)-|(myxor1.in 2)
        ;
        \node[left]at(-5,2.25){$A$};
        \node[left]at(-5,1.75){$B$};
        \node[right]at(0,2){$A.B \ (Carry) $};
        \node[right]at(0,0){$A \oplus B \ (Sum)$};
        
        
    \end{circuitikz}
   
    \caption{Circuit Diagram of Half Adder}
    
\end{figure}
\subsection{Implementation of Half Adder using NAND Gate}
\begin{figure}[h!]
\centering
\resizebox{\columnwidth}{!}
    {
    


\tikzset{every picture/.style={line width=0.75pt}} %set default line width to 0.75pt        

\begin{tikzpicture}[x=0.75pt,y=0.75pt,yscale=-1,xscale=1]
%uncomment if require: \path (0,359); %set diagram left start at 0, and has height of 359

%Flowchart: Delay [id:dp7867230667788392] 
\draw   (135,121) -- (170,121) .. controls (189.33,121) and (205,136.67) .. (205,156) .. controls (205,175.33) and (189.33,191) .. (170,191) -- (135,191) -- cycle ;
%Shape: Circle [id:dp38404098430005407] 
\draw   (205,156) .. controls (205,152.13) and (208.13,149) .. (212,149) .. controls (215.87,149) and (219,152.13) .. (219,156) .. controls (219,159.87) and (215.87,163) .. (212,163) .. controls (208.13,163) and (205,159.87) .. (205,156) -- cycle ;

%Straight Lines [id:da6149641514696975] 
\draw    (135.56,140) -- (57.56,140) ;
%Straight Lines [id:da8187420584765419] 
\draw    (135.56,170) -- (59.56,170) ;
%Flowchart: Delay [id:dp9591336365124401] 
\draw   (281,6) -- (316,6) .. controls (335.33,6) and (351,21.67) .. (351,41) .. controls (351,60.33) and (335.33,76) .. (316,76) -- (281,76) -- cycle ;
%Shape: Circle [id:dp6279703671836241] 
\draw   (351,41) .. controls (351,37.13) and (354.13,34) .. (358,34) .. controls (361.87,34) and (365,37.13) .. (365,41) .. controls (365,44.87) and (361.87,48) .. (358,48) .. controls (354.13,48) and (351,44.87) .. (351,41) -- cycle ;

%Flowchart: Delay [id:dp7304588849669185] 
\draw   (281,211) -- (316,211) .. controls (335.33,211) and (351,226.67) .. (351,246) .. controls (351,265.33) and (335.33,281) .. (316,281) -- (281,281) -- cycle ;
%Shape: Circle [id:dp3534254754649777] 
\draw   (351,246) .. controls (351,242.13) and (354.13,239) .. (358,239) .. controls (361.87,239) and (365,242.13) .. (365,246) .. controls (365,249.87) and (361.87,253) .. (358,253) .. controls (354.13,253) and (351,249.87) .. (351,246) -- cycle ;

%Flowchart: Delay [id:dp1088226222209836] 
\draw   (450,120) -- (485,120) .. controls (504.33,120) and (520,135.67) .. (520,155) .. controls (520,174.33) and (504.33,190) .. (485,190) -- (450,190) -- cycle ;
%Shape: Circle [id:dp5770012512436369] 
\draw   (520,155) .. controls (520,151.13) and (523.13,148) .. (527,148) .. controls (530.87,148) and (534,151.13) .. (534,155) .. controls (534,158.87) and (530.87,162) .. (527,162) .. controls (523.13,162) and (520,158.87) .. (520,155) -- cycle ;

%Straight Lines [id:da6626817599432047] 
\draw    (281.56,26) -- (80.56,26) ;
%Straight Lines [id:da7848454192126346] 
\draw    (80.56,26) -- (80.56,140) ;
%Straight Lines [id:da1980650382342899] 
\draw    (281.56,262) -- (80.56,262) ;
%Straight Lines [id:da7152076447682398] 
\draw    (80.56,170) -- (80.56,262) ;
%Straight Lines [id:da004217873674590145] 
\draw    (280.56,57) -- (242,57) ;
%Straight Lines [id:da6679270780402808] 
\draw    (242,57) -- (242,230) ;
%Straight Lines [id:da22369107623988715] 
\draw    (281,230) -- (242,230) ;
%Straight Lines [id:da34826898750574475] 
\draw    (219,156) -- (242,156) ;
%Straight Lines [id:da4649370106182904] 
\draw    (450,140) -- (404,140) ;
%Straight Lines [id:da30834332610271775] 
\draw    (404,41) -- (365,41) ;
%Straight Lines [id:da29962283531435174] 
\draw    (404,41) -- (404,140) ;
%Straight Lines [id:da8716259082231201] 
\draw    (404,246) -- (365,246) ;
%Straight Lines [id:da8250755525773155] 
\draw    (404,169.6) -- (404,246) ;
%Straight Lines [id:da09084208776438807] 
\draw    (450,169.6) -- (404,169.6) ;
%Straight Lines [id:da47840107613920635] 
\draw    (534,155) -- (575,155) ;
%Flowchart: Delay [id:dp21381644594873905] 
\draw   (451,280) -- (486,280) .. controls (505.33,280) and (521,295.67) .. (521,315) .. controls (521,334.33) and (505.33,350) .. (486,350) -- (451,350) -- cycle ;
%Shape: Circle [id:dp6054967003602114] 
\draw   (521,315) .. controls (521,311.13) and (524.13,308) .. (528,308) .. controls (531.87,308) and (535,311.13) .. (535,315) .. controls (535,318.87) and (531.87,322) .. (528,322) .. controls (524.13,322) and (521,318.87) .. (521,315) -- cycle ;

%Straight Lines [id:da34947422174053977] 
\draw    (451,299) -- (412,299) ;
%Straight Lines [id:da4341194275937226] 
\draw    (451,330) -- (412,330) ;
%Straight Lines [id:da8448460432961056] 
\draw    (412,299) -- (412,330) ;
%Straight Lines [id:da6835099524776336] 
\draw    (230.5,314.5) -- (412,314.5) ;
%Straight Lines [id:da5639417564661686] 
\draw    (230.5,156) -- (230.5,314.5) ;
%Straight Lines [id:da47936892569318656] 
\draw    (535,315) -- (573,315) ;
%Shape: Circle [id:dp09326984519289461] 
\draw  [fill={rgb, 255:red, 0; green, 0; blue, 0 }  ,fill opacity=1 ] (227,156) .. controls (227,154.07) and (228.57,152.5) .. (230.5,152.5) .. controls (232.43,152.5) and (234,154.07) .. (234,156) .. controls (234,157.93) and (232.43,159.5) .. (230.5,159.5) .. controls (228.57,159.5) and (227,157.93) .. (227,156) -- cycle ;
%Shape: Circle [id:dp42002187740887886] 
\draw  [fill={rgb, 255:red, 0; green, 0; blue, 0 }  ,fill opacity=1 ] (238.5,156) .. controls (238.5,154.07) and (240.07,152.5) .. (242,152.5) .. controls (243.93,152.5) and (245.5,154.07) .. (245.5,156) .. controls (245.5,157.93) and (243.93,159.5) .. (242,159.5) .. controls (240.07,159.5) and (238.5,157.93) .. (238.5,156) -- cycle ;
%Shape: Circle [id:dp26554165770879967] 
\draw  [fill={rgb, 255:red, 0; green, 0; blue, 0 }  ,fill opacity=1 ] (77.06,140) .. controls (77.06,138.07) and (78.63,136.5) .. (80.56,136.5) .. controls (82.49,136.5) and (84.06,138.07) .. (84.06,140) .. controls (84.06,141.93) and (82.49,143.5) .. (80.56,143.5) .. controls (78.63,143.5) and (77.06,141.93) .. (77.06,140) -- cycle ;
%Shape: Circle [id:dp3397630763034434] 
\draw  [fill={rgb, 255:red, 0; green, 0; blue, 0 }  ,fill opacity=1 ] (77.06,170) .. controls (77.06,168.07) and (78.63,166.5) .. (80.56,166.5) .. controls (82.49,166.5) and (84.06,168.07) .. (84.06,170) .. controls (84.06,171.93) and (82.49,173.5) .. (80.56,173.5) .. controls (78.63,173.5) and (77.06,171.93) .. (77.06,170) -- cycle ;
%Shape: Circle [id:dp6215419073684494] 
\draw  [fill={rgb, 255:red, 0; green, 0; blue, 0 }  ,fill opacity=1 ] (408.5,314.5) .. controls (408.5,312.57) and (410.07,311) .. (412,311) .. controls (413.93,311) and (415.5,312.57) .. (415.5,314.5) .. controls (415.5,316.43) and (413.93,318) .. (412,318) .. controls (410.07,318) and (408.5,316.43) .. (408.5,314.5) -- cycle ;

% Text Node
\draw (40,129) node [anchor=north west][inner sep=0.75pt]   [align=left] {A};
% Text Node
\draw (41,159) node [anchor=north west][inner sep=0.75pt]   [align=left] {B};
% Text Node
\draw (539,131) node [anchor=north west][inner sep=0.75pt]   [align=left] {Sum};
% Text Node
\draw (535,288) node [anchor=north west][inner sep=0.75pt]   [align=left] {Carry};


\end{tikzpicture}

    }
\caption{Circuit Diagram of Half Adder using NAND Gate}
\end{figure}

\end{document}
