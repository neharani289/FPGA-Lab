\documentclass[journal,12pt,twocolumn]{IEEEtran}
\usepackage[utf8]{inputenc}
\usepackage{amssymb}
\usepackage{setspace}
\usepackage{gensymb}
\singlespacing
\usepackage[mathscr]{euscript}
\usepackage{textgreek}
\usepackage{textcomp}
\usepackage{amsmath}
\numberwithin{equation}{section}
\usepackage{mathrsfs}
\usepackage{txfonts}
\usepackage{stfloats}
\usepackage{bm}
\usepackage{cite}
\usepackage{hyperref}
\usepackage{cases}
\usepackage{subfig}
\usepackage{graphicx}
\usepackage{circuitikz}
\usepackage{tikz} 
\usetikzlibrary{karnaugh}
\usepackage{listings}
\usepackage{amsfonts}
\usepackage{longtable}
\usepackage{multirow}
\usepackage{tcolorbox}
\usepackage{geometry}
\geometry{
 a4paper,
 total={170mm,257mm},
 left=20mm,
 top=20mm,
 }
\usepackage{listings}
\lstset{
frame=single, 
breaklines=true,
columns=fullflexible
}

\def\putbox#1#2#3{\makebox[0in][l]{\makebox[#1][l]{}\raisebox{\baselineskip}[0in][0in]{\raisebox{#2}[0in][0in]{#3}}}}
     \def\rightbox#1{\makebox[0in][r]{#1}}
     \def\centbox#1{\makebox[0in]{#1}}
     \def\topbox#1{\raisebox{-\baselineskip}[0in][0in]{#1}}
     \def\midbox#1{\raisebox{-0.5\baselineskip}[0in][0in]{#1}}
\vspace{3cm}

\usepackage{karnaugh-map}
\usepackage{hyperref}

\renewcommand\thesection{\arabic{section}}
\renewcommand\thesubsection{\thesection.\arabic{subsection}}
\renewcommand\thesubsubsection{\thesubsection.\arabic{subsubsection}}

\renewcommand\thesectiondis{\arabic{section}}
\renewcommand\thesubsectiondis{\thesectiondis.\arabic{subsection}}
\renewcommand\thesubsubsectiondis{\thesubsectiondis.\arabic{subsubsection}}

\title{Assignment 1}
\author{EE20MTECH14014 \\ Neha Rani}
\date{December 2021}

\begin{document}

\maketitle
Download the LaTex code from:
\begin{lstlisting}
https://github.com/neharani289/FPGA-Lab/tree/main/Assignment1
\end{lstlisting}

\section{Question}
\vspace{15pt}

[ICSE 2017 Q5 (b)]\\
State the application of Half Adder. Draw the truth table and circuit diagram for a Half Adder


\section{Solution}
\subsection{Application of Half Adder}
\begin{enumerate}
    \item Half Adder is a combinational logic circuit.
It is used for the purpose of adding two single bit numbers.\\
\item Half adder is used to make full adder as a full adder requires 3 inputs, the third input being an input carry i.e. we will be able to cascade the carry bit from one adder to the other.\\
\item The ALU (arithmetic logic circuitry) of a computer uses half adder to compute the binary addition operation on two bits.\\
\end{enumerate}

\subsection{Truth Table of Half Adder}
\space
The truth table corresponding to sum and carry for various choices of input A, B is shown below:\\
\medskip

    \centering
          \begin{tabular}{|c|c|c|c|}
            \hline
            A & B & Sum & Carry \\
            \hline
            0 & 0 & 0 & 0\\
            \hline
             0 & 1 & 1 & 0\\
            \hline
             1 & 0 & 1 & 0\\
            \hline
             1 & 1 & 0 & 1\\
            \hline
        \end{tabular}
        \vspace{100pt}
\subsection {Circuit Diagram}

\begin{figure}[h!]
    \centering
    \begin{circuitikz}\draw
       
         (0,0) node[xor port](myxor1){}
        (0,2) node[and port](myand1){}
        (-5,2.25) to[short,-*](-3,2.25)
        (-5,1.75) to[short ,-*](-4,1.75)
        (-3,2.25)-|(myand1.in 1)
        (-4,1.75)-|(myand1.in 2)
        (-3,2.25)--(-3,0.25)-|(myxor1.in 1)
        (-4,1.75)--(-4,-0.25)-|(myxor1.in 2)
        ;
        \node[left]at(-5,2.25){$A$};
        \node[left]at(-5,1.75){$B$};
        \node[right]at(0,2){$A.B \ (Carry) $};
        \node[right]at(0,0){$A \oplus B \ (Sum)$};
        
        
    \end{circuitikz}
   
    \caption{Circuit Diagram of Half Adder}
    
\end{figure}

\end{document}
